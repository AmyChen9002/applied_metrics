\documentclass[11pt]{article}
%\usepackage[latin9]{inputenc}
\usepackage{geometry}
\geometry{verbose}
\setlength{\parskip}{6pt}
\setlength{\parindent}{0pt}
\usepackage{color}
\usepackage{amsmath, amssymb}
\usepackage{ dsfont }
\usepackage[authoryear]{natbib}
\usepackage[outputdir=./,cache=false]{minted}
\usepackage[unicode=true,
 bookmarks=false,
 breaklinks=false,pdfborder={0 0 1},backref=section,colorlinks=false]
 {hyperref}
\usepackage{breakurl}

\makeatletter

%%%%%%%%%%%%%%%%%%%%%%%%%%%%%% LyX specific LaTeX commands.
\newcommand{\lyxmathsym}[1]{\ifmmode\begingroup\def\b@ld{bold}
  \text{\ifx\math@version\b@ld\bfseries\fi#1}\endgroup\else#1\fi}
\newcommand{\R}{\mathbb{R}}


%% Because html converters don't know tabularnewline
\providecommand{\tabularnewline}{\\}

%%%%%%%%%%%%%%%%%%%%%%%%%%%%%% User specified LaTeX commands.


%%%%%%%%%%%%%%%%%%%%%%%%%%%%%%%%%%%%%%%%%%%%%%%%%%%%%%%%%%%%%%%%%%%%%%%%%%%%%%%%%%%%%%%%%%%%%%%%%%%%%%%%%%%%%%%%%%%%%%%%%%%%%%%%%%%%%%%%%%%%%%%%%%%%%%%%%%%%%%%%%%%%%%%%%%%%%%%%%%%%%%%%%%%%%%%%%%%%%%%%%%%%%%%%%%%%%%%%%%%%%%%%%%%%%%%%%%%%%%%%%%%%%%%%%%%%
\usepackage{graphicx}
\usepackage{float}
\usepackage{url}
\usepackage{array}
\usepackage{enumitem}
\newcolumntype{L}[1]{>{\raggedright\let\newline\\\arraybackslash\hspace{0pt}}m{#1}}

\title{\Huge Problem Set 4}
\author{\Large Chris Conlon}
\date{\Large Spring 2023}
\makeatother

\begin{document}

\title{Problem Set 4}
\maketitle
\begin{center}
\begin{tabular*}{0.9\textwidth}{@{\extracolsep{\fill}}@{\extracolsep{\fill}}l@{\extracolsep{\fill}}l@{\extracolsep{\fill}}l}
Econometrics I & $\qquad$ & Professor Chris Conlon\tabularnewline
NYU Stern &  & Email: \href{mailto:ctc5@stern.nyu.edu}{ctc5@stern.nyu.edu}\tabularnewline
\end{tabular*}
\par\end{center}


\subsection*{1 Describing Complier Groups}

This problem asks you to analyze Abadie’s (2003) $\chi$-function for describing complier groups. Consider a potential outcomes model for treatment effects with imperfect compliance where $Z_i$ a binary instrumental variable which is 1 if individual $i$ is selected as part of the control group and 0 else, and unit i’s treatment status is described by $$D_i = (1 - Z_i)D_{0i} + Z_i D_{1i},$$ where $D_{0i}$, $D_{1i}$ are the potential values of the treatment indicator for the respective values of $Z_i$. We also observe the exogenous characteristics $X_i$ for each unit and an outcome variable $Y_i = (1 - D_i)Y_{0i} + D_i Y_{1i}$.

\bigskip
\begin{enumerate}[label=\alph*)]

    \item Tabulate the possible combinations for values of the potential treatments ($D_{0i}$, $D_{1i}$), and assign the labels of ”never-takers”, ”always-takers”, ”compliers”, and ”defiers.” What is the role of those treatment groups in the interpretation of the 2SLS estimator for a regression of $Y_i$ on $D_i$ using $Z_i$ as an instrumental variable?

    \item Now we are interested in characterizing the treatment groups in terms of socio-economic background $X_i$ (e.g. parents’ education). Show that $E[X_i]$ is equal to a sum of the conditional expectations 
    \[
    \begin{array}{cl}
    h_{00} =& E[X_i|D_{0i} = 0, D_{1i} = 0] \\
    h_{01} =& E[X_i|D_{0i} = 0, D_{1i} = 1] \\
    h_{10} =& E[X_i|D_{0i} = 1, D_{1i} = 0] \\
    h_{11} =& E[X_i|D_{0i} = 1, D_{1i} = 1] \\
    \end{array}
    \]
     weighted by the probabilities $\pi_{01} := P(D_{0i} = 0, D_{1i} = 1)$, etc..

    \item Suppose the monotonicity condition holds, i.e. $D_{1i} \geq D_{0i}$ for all individuals, and that ($X_i$, $D_i$) is independent of $Z_i$. Show that $$E[D_i(1 - Z_i)X_i] = E[X_i|D_{0i} = 1, Z_i = 0] P(D_{0i} = 1|Z_i = 0) P(Z_i = 0).$$ 

    \item Show that under these conditions $$E[D_i(1 - Z_i)X_i] = E[X_i|D_{0i} = 1, D_{1i} =1]P(D_{0i} = 1, D_{1i} = 1)P(Z_i = 0).$$

    \item Show that under the same conditions $$E[(1 - D_i)Z_i X_i] = E[X_i|D_{0i} = 0, D_{1i}= 0]P(D_{0i} = 0, D_{1i} = 0)P(Z_i = 1).$$

    \item Use your results from parts (b), (d), and (e) to show that for the function $$w_i := 1 - \frac{D_i(1 - Z_i)}{P (Z_i=0)} -  \frac{(1-D_i)Z_i }{P (Z_i=1)}$$ we have $$E[w_i X_i] = E[X_i|D_{0i} = 0, D_{1i} = 1]P(D_{0i} = 0, D_{1i} = 1)$$ How is that result useful to describe the complier group with respect to the instrumental variable $Z_i$ in terms of other observable characteristics?

\end{enumerate}



\subsection*{2 LATE empirically}
%https://mixtape.scunning.com/04-potential_outcomes#steps-to-a-p-value
%http://perseus.iies.su.se/~kburc/teaching/Ec402LTps2_solutions.pdf
% https://gist.github.com/BioSciEconomist/a72fae6e01053fdb6d13c9a80d8e39f9
Download the provided dataset. It is simulated data based on the STAR class size experiment in Tennessee which attempted to estimate the effect of class size on test scores. Treatment (randomly assigned) is initially being put in a small class but students did not necessarily stay in their assigned classroom.
%We will use the notation $d_i(z_i)$ denote whether subject i complies with their treatment status when assigned to  $z_i$.

%So $d_i(0) = 1$  denotes defiers who received treatment when assigned to the control,  $d_i(1) = 0$ are defiers who did not receive treatment when assigned to treatment and $d_i(1) = 1$ and $d_i(0) = 0$ are compliers.
% https://stats.stackexchange.com/questions/11677/what-is-the-difference-between-itt-and-ate
\begin{enumerate}[label=\alph*)]
    \item Who are the compliers, always takers, and never takers and who are defiers in this experiment? What would be the concern with the always takers in this case and the expected impact on results?
    \item Estimate the LATE in your dataset. Estimate the ATE. Estimate the OLS equation. Why do the above differ?
    %\item Now instead estimate the IV treatment effect, what type of effect does it recover.
    \item What is the TOT in this case? When if ever will it differ from LATE? Interpret.
    \item When might it be useful to look at the LATE and when might it be useful to consider TOT? What is more useful in this case?
    \item What is the (theoretically \& in this data) is the difference between ITT and ATE?
\end{enumerate}

\subsection*{3 Regression Discontinuity}
Consider the RDD model with a structural equation of interest
$$y_i = \beta_0  + x_i \beta_1 + w_i \beta_2 + h^1(z_i) + \epsilon_i$$

Where $x$ is the treatment variable and $w$ is a vector of covariates. The first stage regression (how does treatment change at the discontinuity) is of the form 

$$x_i = \pi_0 + D_i \pi_1 + w_i \pi_2 + h^2(z_i) + \nu_i$$

Where $D_i = I(z_i \ge z_0)$ and $\pi_1$ is the coefficient of interest. Assume also that $h^m$ is defined as
$$h^m(z_i) = \sum_{j=1}^\rho \bigg[D_i \delta_j^{m+}(z_i - z_0)^j + (1 - D_i) \delta_j^{m-}(z_i - z_0)^j \bigg]$$

\begin{enumerate}[label=\alph*)]
    \item Show that if we do not subtract off $z_0$ in the term $z_i - z_0$ we cannot treat $\pi_1$ as the impact of crossing the threshold on treatment. To make your life easier, assume linear terms in $z_i - z_0$ only.
\end{enumerate}

\subsection*{4 Regression Discontinuity Replication}
The data is simulated to look similar to the data used by Carpenter and Dobkin (2009), (we can't use the actual data since it is proprietary) they are estimating the effect of alcohol consumption on mortality by utilising the minimum drinking age within a regression discontinuity design. 



\begin{enumerate}[label=\alph*)]
    \item Create a well-labeled scatterplot with mortality on the y-axis and age on the x-axis, mark the minimum drinking age (21)
    \item Create a dummy variable indicating whether an individual is above or below the cutoff, also create a variable that indicates how far (in years) each individual is from the threshold (for both 20 year olds and 22 year olds, this variable should be one). Regress with OLS all deaths per 100,000 on  your  dummy, and your distance to the cutoff. How to interpret your two variables? What constraint does this specification put on the slopes before and after the cutoff?
    \item Now add an interaction between your threshold variable and your distance from threshold variable. How do you interpret the coefficient on this interaction?
    \item Add your regression lines to your scatterplot in part 1.
    \item Now lets use the package ``rdd" in R. Adapting the command below, perform an rdd.
    \begin{minted}{r}
    rdd_model <- RDestimate(formula,
    data, 
    cutpoint = NULL,
    kernel = "triangular")
    \end{minted}
    \item What is the difference between the regression you ran above and the rdd you performed? 
\end{enumerate}




\subsection*{5 Synthetic Controls}
We are going to estimate a synthetic control model here using the package "gsynth" by Xiu (2017) using data from  Abadie (2021) and Abadie, Diamond, and Hainmueller (2015) (hint: to import you will need to use the package "haven" to convert stata files to R). The intervention in this case is the reunification of Germany, the "treated" unit is the former West Germany and the "donor pool" is a set of industrialized countries and your dependent variable is 

%https://gist.github.com/BioSciEconomist/501fd693c26d2b1796eda35929804cb9
% https://pubs.aeaweb.org/doi/pdfplus/10.1257/jel.20191450
% https://rpubs.com/danilofreire/synth
\begin{enumerate}[label=\alph*)]
    \item Graph GDP over time for West Germany and a simple average of GDP for the control countries.
    \item Estimate the change in GDP from reunification using a differences in differences method and the synthetic control model. Your code should look something like this:
    \begin{minted}{R}
    gsynth.out <- gsynth(Y ~ T + X1 + X2, 
                     data = df,
                     index = c("unit","time"), 
                     force = "two-way", 
                     se = TRUE, 
                     inference = "parametric",
                     nboots = 1000,
                     parallel = TRUE)
    \end{minted}
    Where index specifies the structure of your data,  "unit" and "time" should be replaced with the relevant variables in your dataset. The formula is constructed as usual in R. If you want to know more about the other parameters check out the package tutorial available \href{https://yiqingxu.org/packages/gsynth/articles/tutorial.html}{here}.
    \item Graph GDP over time for West Germany and the GDP of the synthetic control, including confidence intervals for your estimates. (The package will do this for you, check out the tutorial linked above for the different plotting options)
    \item What 5 countries are weighted the highest in the synthetic control and what are the weights? One threat to validity is "interference" that is the treatment can spread from the treated unit to the control unit. Why is this a threat in this identification strategy? Is there potential for concern here? Why or why not?
\end{enumerate}


\end{document}
