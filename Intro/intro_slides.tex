\documentclass[aspectratio=169]{beamer}
\usetheme{metropolis}

\usepackage{amsmath,amsfonts}
\usepackage{amsthm}
\usepackage{amssymb}
\usepackage{latexsym}
\usepackage{graphicx}
\usepackage{fancybox}
\usepackage{dsfont}
\usepackage{multirow} 
\usepackage{multicol}
\usepackage{booktabs} 
\usepackage{dcolumn}
%\usepackage[cache=false]{minted}
\usepackage{MnSymbol}
\usepackage{stmaryrd}


\DeclareMathOperator*{\argmax}{arg\,max}
\DeclareMathOperator*{\argmin}{arg\,min}

\newcommand{\X}{\mathtt{X}}
\newcommand{\Y}{\mathtt{Y}}

%\newcommand{\R}{\mathbb{R}}
%\newcommand{\E}{\mathbb{E}}
%\newcommand{\V}{\mathbb{V}}
\newcommand{\p}{\mathbb{P}}
\newcommand*\df{\mathop{}\!\mathrm{d}}
\newcommand{\del}{\partial}


% imports
\usepackage{xargs}
\usepackage{xpatch}
\usepackage{etoolbox}
\usepackage{pdflscape}
\usepackage{booktabs}
\usepackage[skip=0.2\baselineskip]{caption}

% command for inputting raw latex
\makeatletter
\newcommand\primitiveinput[1]{\@@input #1 }
\makeatother




% \usepackage{slashbox}
\title{Lecture 0: Intro}
\author{Chris Conlon }
\institute{NYU Stern }


\newcommand{\norm}[1]{\left\lVert#1\right\rVert}
\newcommand{\R}{\mathbb{R}}
\newcommand{\E}{\mathbb{E}}
\newcommand{\V}{\mathbb{V}}
\newcommand{\ol}{\overline}
%\newcommand{\ul}{\underline}
\newcommand{\pp}{{\prime \prime}}
\newcommand{\ppp}{{\prime \prime \prime}}
\newcommand{\policy}{\gamma}


\newcommand{\fp}{\frame[plain]}

\date{\today}

\begin{document}
\maketitle

\begin{frame}[fragile]{Course Objectives}
\begin{itemize}
\item Review Syllabus
\item I will try and give an overview of a number of different methods, tools, and techniques
\item I will try to teach the course in a very \alert{applied} manner.
\begin{itemize}
\item Will vary lecture to lecture on how much theory vs. application
\end{itemize} 
\item This is a small class- so what matters is what is useful to \alert{you} not to me.
\end{itemize} 
\end{frame}


\begin{frame}[fragile]{Course Organization}
\begin{itemize}
\item My goal is to teach this \alert{flipped}.
\begin{itemize}
\item Watch about 90 minutes of video before each class
\item Come with questions about material
\item Work on exercises in class.
\end{itemize}

\item This is a small class-- please ask plenty of questions
\item If you are lost - don't panic - so is everyone else!
\item I will post slides and code examples to Github on the day of class
\end{itemize} 
\end{frame}

\begin{frame}[fragile]{Grading}
This is a Ph.D. course so you should be done worrying about grades by now
\begin{itemize}
\item (6) Problem Sets
\item (1) Exam / Final Project
\item Class Participation
\end{itemize} 
\end{frame}


\begin{frame}[fragile]{Textbooks}
You should learn from as many sources as possible\\
I will follow two main textbooks.
\begin{itemize}
\item Greene (2017). \textit{Econometric Analysis}. ISBN: 0134461363
\item Tibshirani, Hastie, Friedman (2016), \textit{The Elements of Statistical Learning}. ISBN: 0387848576. Available online at \url{https://web.stanford.edu/~hastie/Papers/ESLII.pdf}.
\item The standard Econometrics Ph.D. course is a bit more theoretical: \url{https://www.ssc.wisc.edu/~bhansen/econometrics/}
\end{itemize} 
\end{frame}


\begin{frame}[fragile]{Software/R stuff}
\texttt{R} is basically two different languages
\begin{itemize}
\item Base \texttt{R} is derived from \texttt{S/S-Plus}
\item ``Modern'' \texttt{R} lives in \texttt{tidyverse} with a very different structure
\item I will jump back and forth as necessary, but I will try to present things from \texttt{tidyverse} when possible
\item Most of my day-to-day is in \texttt{Python} so I am far from an expert in R.
\end{itemize}
\begin{minted}{R}
#always
library(tidyverse)
\end{minted}
\end{frame}


\section*{Thanks!}

\end{document}