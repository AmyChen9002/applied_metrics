\documentclass[handout, serif, aspectratio=169, 10pt]{beamer}

% packages
%\usepackage{newpxmath} % math font is Palatino compatible
%\usepackage[nomath]{fontspec}

\usepackage{setspace}
\usepackage{xcolor}
\usepackage{soul} % for \st
\usepackage{hyperref} % for links
\definecolor{links}{HTML}{2A1B81}
\hypersetup{colorlinks,linkcolor=,urlcolor=links}


% table stuff
\usepackage{chronosys}
\usepackage{verbatim}
% \pagenumbering{arabic}
\usepackage{tabularx}
\usepackage{booktabs}
\usepackage{ragged2e}
\usepackage{mathtools}

% R Code
\usepackage{listings}
\usepackage{courier}
\lstset{basicstyle=\scriptsize\ttfamily,breaklines=true}
\lstset{framextopmargin=50pt,frame=bottomline}

% themes
\usetheme[progressbar=frametitle, block=fill]{metropolis}
\useoutertheme{metropolis}
\useinnertheme{metropolis}

% colors
\definecolor{dimwhite}{rgb}{0.99, 0.99, 0.99}
\definecolor{charcoal}{rgb}{0.21, 0.27, 0.31}
\definecolor{slategray}{rgb}{0.44, 0.5, 0.56}
\definecolor{dimgray}{rgb}{0.41, 0.41, 0.41}
\definecolor{bleudefrance}{rgb}{0.19, 0.55, 0.91}

% beamer options
\setbeamercolor{author}{fg=charcoal}
\setbeamercolor{background canvas}{bg=white}
\setbeamercolor{section in toc}{fg=charcoal}
\setbeamercolor{subsection in toc}{fg=dimgray}
\setbeamercolor{frametitle}{bg=dimwhite, fg=charcoal}
\setbeamercolor{progress bar}{fg=slategray, bg=fg!50!black!30}
\setbeamercovered{transparent}
\setbeamertemplate{itemize items}[triangle]
\setbeamertemplate{itemize subitem}[circle]
\setbeamertemplate{itemize subsubitem}[square]
\setbeamersize{text margin left=7mm,text margin right=7mm} 

% new commands
\newcommand{\q}[1]{``#1''}
\newcommand{\hs}[1]{\textsc{\hfill\scriptsize\color{dimgray}#1}}
\newcommand{\g}[1]{{\color{gray}#1}}
\newcommand{\dg}[1]{{\color{dimgray}#1}}
\newcommand{\sg}[1]{{\color{slategray}#1}}
\newcommand{\bdf}[1]{{\color{bleudefrance}#1}}
\newcommand{\itemcolor}[1]{\renewcommand{\makelabel}[1]{\color{#1}\hfil ##1}}
\newcommand\Wider[2][2em]{
\makebox[\linewidth][c]{
  \begin{minipage}{\dimexpr\textwidth+#1\relax}
  \raggedright#2
  \end{minipage}
  }
}

% misc
\linespread{1.35}

% Math stuff
\newcommand{\norm}[1]{\left\lVert#1\right\rVert}
\newcommand{\R}{\mathbb{R}}
\newcommand{\E}{\mathbb{E}}
\newcommand{\V}{\mathbb{V}}
\newcommand{\probP}{\mathbb{P}}
\newcommand{\ol}{\overline}
%\newcommand{\ul}{\underline}
\newcommand{\pp}{{\prime \prime}}
\newcommand{\ppp}{{\prime \prime \prime}}
\newcommand{\policy}{\gamma}
\newcommand{\plim}{ \overset{p}{\to}}
\newcommand{\hnot}{ \overset{H_0}{\to}}

% Causal Graphs
\usetikzlibrary{shapes,decorations,arrows,calc,arrows.meta,fit,positioning}
\tikzset{
    -Latex,auto,node distance =1 cm and 1 cm,semithick,
    state/.style ={ellipse, draw, minimum width = 0.7 cm},
    point/.style = {circle, draw, inner sep=0.04cm,fill,node contents={}},
    bidirected/.style={Latex-Latex,dashed},
    el/.style = {inner sep=2pt, align=left, sloped}
}


% \usepackage{slashbox}
\title{ Double/Debiased Machine Learning}
\author{Chris Conlon }
\institute{NYU Stern }



\date{\today}

\begin{document}
\maketitle

\section*{Setup}

\begin{frame}{The core problem}
\begin{align*}
Y_i &= \alert{\tau_i} \cdot D_i + g_0(X_i) + u_i &\text{ with }\E[u_i \mid x_i, D_i]=0\\
D_i &= m_0(X_i) + v_i &\text{ with } \E[v_i \mid x_i ]=0
\end{align*}
\begin{itemize}
    \item $D_i$ is \alert{treatment indicator} and $\tau_i$  is \alert{treatment effect}
    \item $X_i$ are covariates (``controls'' or ``confounders'') $\rightarrow$
    \item We call $m_0(\cdot)$ and $g_0(\cdot)$ \alert{nuisance parameters} or \alert{nuisance functions}.
\end{itemize}
\end{frame}

\begin{frame}{What if we try ``machine learning?`'}
\begin{align*}
Y_i &= \alert{\tau_i} \cdot D_i + g_0(X_i) + u_i &\text{ with }\E[u_i \mid x_i, D_i]=0\\
D_i &= m_0(X_i) + v_i &\text{ with } \E[v_i \mid x_i ]=0
\end{align*}
Could alternate steps:
\begin{enumerate}
    \item Fit random forest of $Y_i - \widehat{\tau_i} \cdot D_i$ on $Z_i$ to get $\widehat{g_0}(Z_i)$.
    \item Run OLS of $Y_i - \widehat{g_0}(Z_i)$ on $D_i$ to get $\widehat{\tau}$ or $\widehat{\tau_i}$
\end{enumerate}
This fits the data great but gives terrible estimates of $\widehat{\tau}$!\\
Why? Frisch-Lovell-Waugh really needs things to be \alert{linear}!
\end{frame}

\begin{frame}{What goes wrong}
\begin{itemize}
\item We are trading off \alert{bias} for \alert{variance reduction}
\item But, we can't trust \alert{plug-in} estimates when $m_0(\cdot),g_0(\cdot)$ are not linear.
\item So bias can be very dangerous now...
\end{itemize}
What we need to do is \alert{orthogonalize} things properly (like a nonlinear Frisch-Lovell-Waugh)
\begin{itemize}
\item Bias from \alert{regularization} \textrightarrow Orthogonalization
\item Bias from \alert{overfitting} \textrightarrow Sample Splitting
\end{itemize}
\end{frame}



\begin{frame}{Sample Splitting}

\begin{itemize}
\item Split the sample into two parts \alert{main sample} and \alert{auxiliary sample}.
\item On the auxiliary Sample:
\begin{itemize}
\item Estimate $\widehat{g}(X_i)$ from $Y_i = \tau \cdot D_i + g(X_i) + u_i $
\item Estimate $\widehat{m}(X_i)$ from $D_i =  m(X_i) + v_i $
\end{itemize}
\item Now on the main sample:
\begin{itemize}
\item Compute the residual:  $\widehat{v_i} = D_i - \widehat{m}(X_i)$.
\item Estimate $\hat{\tau}=\left(\hat{v}^{\prime} D\right)^{-1} \hat{v}^{\prime}(Y-\hat{g}(X))$
\end{itemize}
\end{itemize}

\end{frame}

\begin{frame}{More Info}
To learn more watch Chernozhukov lecture here:
\url{https://www.youtube.com/watch?v=eHOjmyoPCFU&t=37s}\\

The \texttt{R} package
\url{https://docs.doubleml.org/stable/index.html}
\end{frame}



\section*{Thanks!}




\end{document}