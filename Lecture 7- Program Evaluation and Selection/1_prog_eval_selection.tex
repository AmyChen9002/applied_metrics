\documentclass[xcolor=pdftex,dvipsnames,table,mathserif,aspectratio=169]{beamer}
\usetheme{default}
\usetheme{metropolis}
\setbeamersize{text margin left=.3in,text margin right=.3in} 

%\usetheme{Darmstadt}
%\usepackage{times}
%\usefonttheme{structurebold}

\usepackage[english]{babel}
%\usepackage[table]{xcolor}
\usepackage{pgf,pgfarrows,pgfnodes,pgfautomata,pgfheaps}
\usepackage{amsmath,amssymb,setspace,centernot}
\usepackage[latin1]{inputenc}
\usepackage[T1]{fontenc}
\usepackage{relsize}
\usepackage{pdfpages}
\usepackage[absolute,overlay]{textpos} 


\newenvironment{reference}[2]{% 
  \begin{textblock*}{\textwidth}(#1,#2) 
      \footnotesize\it\bgroup\color{red!50!black}}{\egroup\end{textblock*}} 

\DeclareMathSizes{10}{10}{6}{6} 

\begin{document}
\title{Program Evaluation(a):
The Selection Problem}
\author{Chris Conlon}
\institute{Applied Econometrics}
\date{\today}

\frame{\titlepage}

\section{Intro}
\frame{\frametitle{Overview}
This set of lectures will cover (roughly) the following papers:\\
Theory:
\begin{itemize}
\item Angrist and Imbens (1994)
\item Heckman Vytlacil (2005/2007)
\item Abadie and Imbens (2006)
\end{itemize}
And draw heavily upon notes by 
\begin{itemize}
\item Guido Imbens
\item Richard Blundell and Costas Meghir
\end{itemize}

}

\begin{frame}
\frametitle{The Evaluation Problem}
\begin{itemize}
\item The issue we are concerned about is identifying the effect of a policy or an investment or some individual action on one or more outcomes of interest
\item This has become the workhorse approach of the applied microeconomics fields (Public, Labor, etc.)
\item Examples may include:
\begin{itemize}
\item The effect of taxes on labor supply
\item The effect of education on wages
\item The effect of incarceration on recidivism
\item The effect of competition between schools on schooling quality
\item The effect of price cap regulation on consumer welfare
\item The effect of indirect taxes on demand
\item The effects of environmental regulation on incomes
\item The effects of labor market regulation and minimum wages on wages and employment
\end{itemize}
\end{itemize}
\end{frame}
\begin{frame}
\frametitle{Example: Borjas (1987)}
\begin{itemize}
\item Consider two countries $(0/1)$ (source and host).
\begin{align*}
\ln w_0 &= \alpha_0 + u_0 \quad \mbox{ with } u_0 \sim N(0,\sigma_0^2) \mbox{ source country}\\
\ln w_1 &= \alpha_1 + u_1 \quad \mbox{ with } u_1 \sim N(0,\sigma_1^2) \mbox{ host country}
\end{align*}
\item Now we allow for migration cost of $C$ which he writes in hours: $\pi = \frac{C}{w_0}$.
\item Assume workers know everything; you only see $u_0$ \alert{OR} $u_1$ depending on country.
\item Correlation in earnings is $\rho=\frac{\sigma_{01}}{\sigma_{0} \sigma_{1}}$.
\end{itemize}
\end{frame}

\begin{frame}
\frametitle{Example: Borjas (1987)}
\begin{itemize}
\item Workers will migrate if:
\begin{align*}
\left( \alpha_{1}-\alpha_{0}-\pi\right)+\left(u_{1}-u_{0}\right)>0
\end{align*}
\item Who migrates? Probability of migration. Define $\nu = u_1-u_0$.
\begin{align*}
P &=\operatorname{Pr}\left[\nu>\left(\alpha_{0}-\alpha_{1}+\pi\right)\right] =\operatorname{Pr}\left[\frac{\nu}{\sigma_{\nu}}>\frac{\left(\alpha_{0}-\alpha_{1}+\pi\right)}{\sigma_{\nu}}\right] \\
&=1-\Phi\left(\frac{\left(\alpha_{0}-\alpha_{1}+\pi\right)}{\sigma_{\nu}}\right) \equiv 1-\Phi(z)
\end{align*}
\item Higher $z$ $\rightarrow$ less migration.
\end{itemize}
\end{frame}


\begin{frame}
\frametitle{Example: Borjas (1987): How does selection work?}
Construct \alert{counterfactual wages} for workers in \alert{source} country for those who immigrate:
\begin{itemize}
\item For now ignore mean differences $\alpha_0 = \alpha_1 = \alpha$.
\begin{align*}
E\left(w_{0} | \text { Immigrate }\right) &=\alpha+E\left(\varepsilon_{0} | \frac{\nu}{\sigma_{\nu}}>z\right) \\
&=\alpha+\sigma_{0} E\left(\frac{\varepsilon_{0}}{\sigma_{0}} | \frac{\nu}{\sigma_{\nu}}>z\right)
\end{align*}
\item Wages depend on:
\begin{enumerate}
\item Mean earnings in the source country
\item Both error terms $\left(u_{0}, u_{1}\right)$ through $\nu$
\item Implicitly, it also depends on the correlation between the error terms.
\end{enumerate}
\end{itemize}
\end{frame}


\begin{frame}
\frametitle{Example: Borjas (1987): How does selection work?}
\begin{itemize}
\item If everything is normal, we just run univariate regression $E\left(u_{0} | \nu\right)=\frac{\sigma_{0 \nu}}{\sigma_{\nu}^{2}} \nu$:
\begin{align*}
E\left(\frac{u_{0}}{\sigma_{0}} | \frac{\nu}{\sigma_{\nu}}\right) &= \frac{1}{\sigma_{0}}\cdot \frac{\sigma_{0 \nu}}{\sigma_{\nu}^{2}} \cdot \frac{\sigma_{\nu}^{2}}{\sigma_{\nu}^{2}} \cdot \nu  
=\frac{\sigma_{0 \nu}}{\sigma_{0} \sigma_{\nu}} \frac{\nu}{\sigma_{\nu}}  =\rho_{0 \nu} \frac{\nu}{\sigma_{\nu}} 
\end{align*}
\begin{align*}
E\left(w_{0} | \text { Immigrate }\right) 
&=\alpha_{0}+\sigma_{0} E\left(\frac{u_{0}}{\sigma_{0}} | \frac{\nu}{\sigma_{\nu}}>z\right) \\
 &=\alpha_{0}+\rho_{0 \nu} \sigma_{0} E\left(\frac{\nu}{\sigma_{\nu}} | \frac{\nu}{\sigma_{\nu}}>z\right) \\
  &=\alpha_{0}+\rho_{0 \nu} \sigma_{0}\left(\frac{\phi(z)}{1-\Phi(z)}\right) 
\end{align*}
\item This hazard rate of the standard normal has a special name \alert{Inverse Mills Ratio} $E[x | x > z]$.
\end{itemize}
\end{frame}


\begin{frame}
\frametitle{Example: Borjas (1987): How does selection work?}
\begin{itemize}
\item A similar expression for those who do immigrate:
\begin{align*}
E\left(w_{1} | \text { Immigrate }\right) &=\alpha_{1}+E\left(u_{1} | \frac{\nu}{\sigma_{\nu}}>z\right) \\
&=\alpha_{1}+\rho_{1 \nu} \sigma_{1}\left(\frac{\phi(z)}{\Phi(-z)}\right)
\end{align*}
\item We can re-write both expressions in terms of the \alert{Inverse Mills Ratio}
\end{itemize}
\end{frame}

\begin{frame}{Inverse Mills Ratio}
\begin{align*}
E\left(w_{0} | \text { Immigrate }\right) &=\alpha_{0}+\rho_{0 \nu} \sigma_{0}\left(\frac{\phi(z)}{1-\Phi(z)}\right) \\
&=\alpha_{0}+\frac{\sigma_{0} \sigma_{1}}{\sigma_{\nu}}\left(\rho-\frac{\sigma_{0}}{\sigma_{1}}\right)\left(\frac{\phi(z)}{1-\Phi(z)}\right) \\
E\left(w_{1} | \text { Immigrate }\right) &=\alpha_{1}+\rho_{1 \nu} \sigma_{1}\left(\frac{\phi(z)}{1-\Phi(z)}\right) \\
&=\alpha_{1}+\frac{\sigma_{0} \sigma_{1}}{\sigma_{\nu}}\left(\frac{\sigma_{1}}{\sigma_{0}}-\rho\right)\left(\frac{\phi(z)}{1-\Phi(z)}\right)
\end{align*}
Where $\rho=\sigma_{01} / \sigma_{0} \sigma_{1}$.
\end{frame}



\begin{frame}{Positive Hierarchical Sorting}
Let $Q_{0}=E\left(u_{0} | I=1\right), Q_{1}=E\left(u_{1} | I=1\right)$ (expected \alert{skill} of immigrants).
\begin{itemize}
\item Immigrants are positively selected and above average $(Q_0,Q_1) > 0$ and $\frac{\sigma_{1}}{\sigma_{0}}>1 \text { and } \rho>\frac{\sigma_{0}}{\sigma_{1}}$
\begin{itemize}
\item $\frac{\sigma_{1}}{\sigma_{0}}>1$ returns to ``skill'' are higher in host country.
\item $\rho>\frac{\sigma_{0}}{\sigma_{1}}$ correlation between valued skills in both counties is high (similar skills valued in both countries).
\end{itemize}
\item Best and brightest leave because returns to skill are too low in home country.
\end{itemize}
\end{frame}

\begin{frame}{Negative Hierarchical Sorting}
We swap the standard deviations:
\begin{itemize}
\item Immigrants are negatively selected and below average $(Q_0,Q_1) < 0$ and $\frac{\sigma_{1}}{\sigma_{0}}>1 \text { and } \rho>\frac{\sigma_{0}}{\sigma_{1}}$
\begin{itemize}
\item $\frac{\sigma_{0}}{\sigma_{1}}>1$ returns to ``skill'' are lower in host country.
\item $\rho>\frac{\sigma_{1}}{\sigma_{0}}$ correlation between valued skills in both counties is high (similar skills valued in both countries).
\end{itemize}
\item Compressed wage structure attracts the low skill types because it provides ``insurance'' or ``subsidizes'' low wage workers.
\end{itemize}
\end{frame}

\begin{frame}{Refugee/Superman Sorting?}
\begin{itemize}
\item Immigrants are below average at home and above average in host $(Q_0<0 ,Q_1>1)$ and $\frac{\sigma_{1}}{\sigma_{0}}>1$:
\begin{itemize}
\item $\rho<\min \left(\frac{\sigma_{1}}{\sigma_{0}}, \frac{\sigma_{0}}{\sigma_{1}}\right)$ being below average in source country makes you above average in host country.
\end{itemize}
\item You are a nerdy intellectual in a country that values physical labor, or are otherwise discriminated against in the labor market.
\end{itemize}
The missing (fourth) case:
\begin{itemize}
\item Mathematically impossible $\rho>\max \left(\frac{\sigma_{1}}{\sigma_{0}}, \frac{\sigma_{0}}{\sigma_{1}}\right)$
\end{itemize}
\end{frame}


\begin{frame}
\frametitle{The Evaluation Problem}
\begin{itemize}
\item Define an outcome variable $Y_i$ for each individual
\item Two \alert{potential outcomes} for each person $\{Y_i(1), Y_i(0)\}$ depending on whether they receive treatment or not.
\item Call $Y_i(1)- Y_i(0) = \beta_i$ the \alert{Treatment effect}.
\item Two major problems:
\begin{itemize}
\item All individuals have different treatment effects (\alert{heterogeneity}).
\item We don't actually observe any one person's treatment effect ! (Missing Data problem)
\end{itemize}
\item We need strong assumptions in order to recover $f(\beta_i)$ from data.
\item Instead we can characterize simpler functions such as $E[\beta_i]$ (ATE) or $E[\beta_i | T_i = 1]$ (ATT) or $E[\beta_i | T_i = 0]$ (ATC)  with fewer restrictions.
\end{itemize}
\end{frame}

\begin{frame}
\frametitle{More Difficulties}
What is hard here?
\begin{itemize}
\item Heterogeneous effect of $\beta_i$ in population.
\item Selection in treatment may be endogenous. That is $T_i$ depends on $Y_i(1),Y_i(0)$.
\item Fisher or Roy (1951) model:
\begin{eqnarray*}
Y_i = (Y_i(1) - Y_i(0)) T_i + Y_i(0)= \alpha + \beta_i T_i + u_i
\end{eqnarray*}
\item Agents usually choose $T_i$ with $\beta_i$ or $u_i$ in mind.
\item Can't necessarily pool across individuals since $\beta_i$ is not constant.
\end{itemize}
\end{frame}




\begin{frame}
\frametitle{Structural vs. Reduced Form}
\begin{itemize}
\item Usually we are interested in one or two parameters of the distribution of $\beta_i$ (such as the average treatment effect or average treatment on the treated).
\item Most program evaluation approaches seek to identify one effect or the other effect. This leads to these as being described as \alert{reduced form} or \alert{quasi-experimental}.
\item The \alert{structural} approach attempts to recover the entire joint $f(\beta_i,u_i)$ distribution but generally requires more assumptions, but then we can calculate whatever we need.
\end{itemize}
\end{frame}

\begin{frame}
\frametitle{Start with Easy Cases}
\begin{itemize}
\item Let's start with the easy cases: run OLS and see what happens.
\item OLS compares mean of treatment group with mean of control group (possibly controlling for other $X$)
\begin{eqnarray*}
\beta^{OLS} &=& E(Y_i | T_i =1) - E(Y_i | T_i=0) \\
&=& \underbrace{E[\beta_i | T_i =1]}_{\mbox{ATT}} + \left(\underbrace{E[u_i | T_i =1 ] - E[u_i | T_i=0] }_{\mbox{selection bias}}  \right)
\end{eqnarray*}
\item Even in absence of heterogeneity $\beta_i = \beta$ we can still have selection bias. 
\item $Y_i^0 = \alpha + u_i$ may vary within the population (this is quite common).
\end{itemize}
\end{frame}
\end{document}
