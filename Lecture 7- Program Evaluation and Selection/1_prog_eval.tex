\documentclass[xcolor=pdftex,dvipsnames,table,mathserif,aspectratio=169]{beamer}
\usetheme{default}
\usetheme{metropolis}
\setbeamersize{text margin left=.3in,text margin right=.3in} 

%\usetheme{Darmstadt}
%\usepackage{times}
%\usefonttheme{structurebold}

\usepackage[english]{babel}
%\usepackage[table]{xcolor}
\usepackage{pgf,pgfarrows,pgfnodes,pgfautomata,pgfheaps}
\usepackage{amsmath,amssymb,setspace,centernot}
\usepackage[latin1]{inputenc}
\usepackage[T1]{fontenc}
\usepackage{relsize}
\usepackage{pdfpages}
\usepackage[absolute,overlay]{textpos} 
\usepackage{dcolumn}
\usepackage{booktabs}

\newenvironment{reference}[2]{% 
  \begin{textblock*}{\textwidth}(#1,#2) 
      \footnotesize\it\bgroup\color{red!50!black}}{\egroup\end{textblock*}} 

\DeclareMathSizes{10}{10}{6}{6} 

\begin{document}
\title{Program Evaluation(a): Intro and Notation}
\author{Chris Conlon}
\institute{Applied Econometrics}
\date{\today}

\frame{\titlepage}

\frame{\frametitle{Overview}
This set of lectures will cover (roughly) the following papers:\\
Theory:
\begin{itemize}
\item Angrist and Imbens (1994)
\item Heckman Vytlacil (2005/2007)
\item Abadie and Imbens (2006)
\end{itemize}
And draw heavily upon notes by 
\begin{itemize}
\item Guido Imbens
\item Richard Blundell and Costas Meghir
\end{itemize}

}

\begin{frame}
\frametitle{The Evaluation Problem}
\begin{itemize}
\item The issue we are concerned about is identifying the effect of a policy or an investment or some individual action on one or more outcomes of interest
\item This has become the workhorse approach of the applied microeconomics fields (Public, Labor, etc.)
\item Examples may include:
\begin{itemize}
\item The effect of taxes on labor supply
\item The effect of education on wages
\item The effect of incarceration on recidivism
\item The effect of competition between schools on schooling quality
\item The effect of price cap regulation on consumer welfare
\item The effect of indirect taxes on demand
\item The effects of environmental regulation on incomes
\item The effects of labor market regulation and minimum wages on wages and employment
\end{itemize}
\end{itemize}
\end{frame}

\begin{frame}{Potential Outcomes}
\begin{itemize}
\item Consider a binary treatment $T_i \in \{0,1\}$. 
\begin{itemize}
\item Some people use $D_i \in \{0,1\}$ instead.
\end{itemize}
\item We observe the outcome $Y_i$. But there are two \alert{potential outcomes}
\begin{itemize}
\item $Y_i(1)$ the outcome for $i$ if they  \alert{are treated}.
\item $Y_i(0)$ the outcome for $i$ if they \alert{are not treated} (control).
\end{itemize}
\item We are generally interested in $\beta_i \equiv  Y_i(1)-Y_i(0)$ which we call the \alert{treatment effect}.
\begin{itemize}
\item Individuals have \alert{heterogeneous treatment effects}.
\item In an ideal world we could fully characterize $f(\beta_i)$
\end{itemize}
\end{itemize}
\end{frame}

\begin{frame}{Some Challenges}
Stable unit treatment value assumption (SUTVA)
\begin{itemize}
\item We assume a \textit{ceteris paribus} version of treatment effects
\item We need $\beta_i$ to be a policy invariant (structural) parameter.
\item Your $\beta_i$ doesn't respond to whether or not another individual is treated.
\item Two common limitations:
\begin{itemize}
\item Peer effects: Whether you respond to job training program depends on whether your spouse is also treated.
\item Equilibrium effects: if we sent everyone to college, returns to college would be quite different.
\end{itemize}
\end{itemize}
\end{frame}



\begin{frame}{Some Challenges}
\begin{columns}[T] % align columns
\begin{column}{.65\textwidth}
Fundamental Problem of Causal Inference
\begin{itemize}
\item We don't observe the \alert{counterfactual} $Y_i(T_i)$.
\item For a single individual we either observe $Y_i(1)$ \alert{or} $Y_i(0)$ but never both!
\begin{itemize}
\item ex: We don't see what your wage would have been if you didn't attend college.
\item ex: We might know your cholesterol before you took Lipitor, but we don't know what it would be today if you didn't take Lipitor.
\end{itemize}
\end{itemize}
\end{column}%
\hfill%
\begin{column}{.38\textwidth}

  \vspace{20pt}
  
  $Y_{i} = T_{i}Y_{i}(1) + (1-T_{i})Y_{i}(0)$\\
  \vspace{20pt}
  \begin{tabular}{ccccc}
    \toprule
    i & $Y_{i}(1)$ &  $Y_{i}(0)$ & $T_{i}$ & $Y_{i}$ \\
    \midrule
    1 &     1      &     \alert{?}      &   1   & 1 \\
    2 &     0      &     \alert{?}      &   1   & 0 \\
    3 &     \alert{?}      &     0      &   0   & 0 \\
     & &  \vdots & & \\
    $n$ &     \alert{?}      &     1      &   0   & 1 \\    
  \end{tabular}
\end{column}%
\end{columns}
\end{frame}


\begin{frame}
\frametitle{Structural vs. Reduced Form}
\begin{itemize}
\item Usually we are interested in one or two parameters of the distribution of $\beta_i$ (such as the average treatment effect or average treatment on the treated).
\item Most program evaluation approaches seek to identify one effect or the other effect. This leads to these as being described as \alert{reduced form} or \alert{quasi-experimental}.
\item The \alert{structural} approach attempts to recover the entire joint $f(\beta_i,u_i)$ distribution but generally requires more assumptions, but then we can calculate whatever we need.
\end{itemize}
\end{frame}


\begin{frame}
\frametitle{Treatment Effects Parameters}
Most approaches to estimating treatment effects will recover some moments of $f(\beta_i)$ instead of the entire distribution
\begin{description}
\item[Average Treatment Effect (ATE)] corresponds to $\mathbb{E}[\beta_i]$.
\item[Average Treatment on Treated (ATT)] corresponds to $\mathbb{E}[\beta_i | T_i = 1]$.
\item[Average Treatment on Control/Untreated (ATUT)] corresponds to $\mathbb{E}[\beta_i | T_i = 0]$.
\end{description}
We also have that if the probability of treatment $ Pr(T_i=1) = \pi$
\begin{align*}
ATE = \pi \cdot ATT + (1-\pi) \cdot ATUT
\end{align*}
\end{frame}

\begin{frame}
\frametitle{Local Average Treatment Effects}
Another important object is the \alert{Wald Estimator}
\begin{align*}
Wald = \frac{\mathbb{E}[Y_i | Z_i =1] - \mathbb{E}[Y_i | Z_i =0]}{\mathbb{E}[T_i  | Z_i =1] - \mathbb{E}[T_i | Z_i =0]} 
\end{align*}
This is useful because under some conditions it corresponds to the 2SLS estimate of $Y_i$ on $T_i$ with binary instrument $Z_i$.
\begin{align*}
Y_i &= \alpha + \beta_i \cdot T_i + u_i \\
T_i &= \lambda + \pi_i \cdot Z_i + e_i
\end{align*}
\end{frame}


\begin{frame}
\frametitle{Intent to Treat}
We can decompose the numerator and the denominator of the \alert{Wald Estimator}
\begin{align*}
Wald = \frac{\mathbb{E}[Y_i | Z_i =1] - \mathbb{E}[Y_i | Z_i =0]}{\mathbb{E}[T_i  | Z_i =1] - \mathbb{E}[T_i | Z_i =0]}  = \frac{ITT}{ITT_d}
\end{align*}
\begin{itemize}
\item \alert{Intent to Treat} (numerator) tells us how outcome responds directly to the instrument.
\item \alert{Intent to Treat ``D''} (denominator) tells us how treatment probability responds directly to the instrument.
\end{itemize}
Often people we report the numerator in addition to other parameters.
\end{frame}


\begin{frame}
\frametitle{Local Average Treatment Effects}
Under conditions we will explore later in detail 2SLS delivers the \alert{local average treatment effect (LATE)}:
\begin{align*}
\widehat{\beta}_1^{TSLS} &\rightarrow^p \frac{\mathbb{E}[\beta_{i} \pi_{i}]}{\mathbb{E}[\pi_{i}]} = LATE \\
LATE &= ATE + \frac{Cov(\beta_{i},\pi_{i})}{\mathbb{E}[\pi_{i}]}
\end{align*}
\begin{itemize}
\item Weighted average for individuals for whom $Z_i$ pushes them into treatment (compliers).
\item Places more weight on individuals with larger $\pi_i$
\item Relationship to ATE depends on correlation between $(\beta_i, \pi_i)$.
\end{itemize}
\end{frame}


\begin{frame}
\frametitle{The Selection Problem}
\begin{itemize}
\item Let's start with the easy cases: run OLS and see what happens.
\begin{align*}
Y_i = \alpha + \beta_i \cdot T_i + u_i
\end{align*}
\item OLS compares mean of treatment group with mean of control group (possibly controlling for other $X$)
\begin{eqnarray*}
\beta^{OLS} &=& E(Y_i | T_i =1) - E(Y_i | T_i=0) \\
&=& \underbrace{E[\beta_i | T_i =1]}_{\mbox{ATT}} + \left(\underbrace{E[u_i | T_i =1 ] - E[u_i | T_i=0] }_{\mbox{selection bias}}  \right)
\end{eqnarray*}
\item Even in absence of heterogeneity $\beta_i = \beta$ we can still have selection bias. 
\item $Y_i^0 = \alpha + u_i$ may vary within the population (this is quite common).
\end{itemize}
\end{frame}

\begin{frame}
\frametitle{Why worry about selection?}
Unless we have random assignment...
\begin{itemize}
\item People often choose $D_i$ with $\beta_i$ in mind.
\item The problem: $D_i \perp u_i$ and/or $D_i \perp \beta_i$ are likely violated.
\item We can get positive or negative selection bias:
\begin{itemize}
\item e.g. Who goes to college? those likely to benefit more than most!
\item e.g. who gets risky surgeries/drugs? people who are very sick.
\end{itemize}
\end{itemize}
\end{frame}

\begin{frame}
\frametitle{What's next?}
Even the simple cases here are pretty tough (Binary treatment, binary (or no) instrument).\\

How do we construct counterfactuals that we don't observe?
\begin{itemize}
\item Matching
\item Regression Adjustment
\item Instrumental Variables
\item Panel Data
\end{itemize}
\end{frame}





\end{document}
