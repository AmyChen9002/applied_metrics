\documentclass[handout, serif, aspectratio=169, 10pt]{beamer}

% packages
%\usepackage{newpxmath} % math font is Palatino compatible
%\usepackage[nomath]{fontspec}

\usepackage{setspace}
\usepackage{xcolor}
\usepackage{soul} % for \st
\usepackage{hyperref} % for links
\definecolor{links}{HTML}{2A1B81}
\hypersetup{colorlinks,linkcolor=,urlcolor=links}


% table stuff
\usepackage{chronosys}
\usepackage{verbatim}
% \pagenumbering{arabic}
\usepackage{tabularx}
\usepackage{booktabs}
\usepackage{ragged2e}
\usepackage{mathtools}

% R Code
\usepackage{listings}
\usepackage{courier}
\lstset{basicstyle=\scriptsize\ttfamily,breaklines=true}
\lstset{framextopmargin=50pt,frame=bottomline}

% themes
\usetheme[progressbar=frametitle, block=fill]{metropolis}
\useoutertheme{metropolis}
\useinnertheme{metropolis}

% colors
\definecolor{dimwhite}{rgb}{0.99, 0.99, 0.99}
\definecolor{charcoal}{rgb}{0.21, 0.27, 0.31}
\definecolor{slategray}{rgb}{0.44, 0.5, 0.56}
\definecolor{dimgray}{rgb}{0.41, 0.41, 0.41}
\definecolor{bleudefrance}{rgb}{0.19, 0.55, 0.91}

% beamer options
\setbeamercolor{author}{fg=charcoal}
\setbeamercolor{background canvas}{bg=white}
\setbeamercolor{section in toc}{fg=charcoal}
\setbeamercolor{subsection in toc}{fg=dimgray}
\setbeamercolor{frametitle}{bg=dimwhite, fg=charcoal}
\setbeamercolor{progress bar}{fg=slategray, bg=fg!50!black!30}
\setbeamercovered{transparent}
\setbeamertemplate{itemize items}[triangle]
\setbeamertemplate{itemize subitem}[circle]
\setbeamertemplate{itemize subsubitem}[square]
\setbeamersize{text margin left=7mm,text margin right=7mm} 

% new commands
\newcommand{\q}[1]{``#1''}
\newcommand{\hs}[1]{\textsc{\hfill\scriptsize\color{dimgray}#1}}
\newcommand{\g}[1]{{\color{gray}#1}}
\newcommand{\dg}[1]{{\color{dimgray}#1}}
\newcommand{\sg}[1]{{\color{slategray}#1}}
\newcommand{\bdf}[1]{{\color{bleudefrance}#1}}
\newcommand{\itemcolor}[1]{\renewcommand{\makelabel}[1]{\color{#1}\hfil ##1}}
\newcommand\Wider[2][2em]{
\makebox[\linewidth][c]{
  \begin{minipage}{\dimexpr\textwidth+#1\relax}
  \raggedright#2
  \end{minipage}
  }
}

% misc
\linespread{1.35}

% Math stuff
\newcommand{\norm}[1]{\left\lVert#1\right\rVert}
\newcommand{\R}{\mathbb{R}}
\newcommand{\E}{\mathbb{E}}
\newcommand{\V}{\mathbb{V}}
\newcommand{\probP}{\mathbb{P}}
\newcommand{\ol}{\overline}
%\newcommand{\ul}{\underline}
\newcommand{\pp}{{\prime \prime}}
\newcommand{\ppp}{{\prime \prime \prime}}
\newcommand{\policy}{\gamma}
\newcommand{\plim}{ \overset{p}{\to}}
\newcommand{\hnot}{ \overset{H_0}{\to}}

% Causal Graphs
\usetikzlibrary{shapes,decorations,arrows,calc,arrows.meta,fit,positioning}
\tikzset{
    -Latex,auto,node distance =1 cm and 1 cm,semithick,
    state/.style ={ellipse, draw, minimum width = 0.7 cm},
    point/.style = {circle, draw, inner sep=0.04cm,fill,node contents={}},
    bidirected/.style={Latex-Latex,dashed},
    el/.style = {inner sep=2pt, align=left, sloped}
}

\title [Nonparametrics]{Nonparametrics and Local Methods: Semiparametrics}
\author{C.Conlon}
\institute{Applied Econometrics}
\date{\today}
\setbeamerfont{equation}{size=\tiny}
\begin{document}

\begin{frame}
\titlepage
\end{frame}


 \frame{\frametitle{The
Seminonparametric Approach}
\begin{itemize}
\item If we are ``pretty sure'' that $f$ is almost $f_{m,\sigma}$ for some
family of densities indexed by  $(m,\sigma)$, then we can choose a family of positive functions of increasing complexity  $P^1_\theta,P^2_\theta,\ldots$
\item Choose some $M$ that goes to infinity as $n$ does (more slowly), and
maximize over $(m,\sigma,\theta)$ the loglikelihood

\[
\sum_{i=1}^n \log f_{m,\sigma}(y_i)P^M_\theta(y_i).
\]
It works\ldots but it is hard to constrain it to be a density for
large $M$. \end{itemize}}



\frame{\frametitle{Mixtures of Normals}

\pause

A special case of seminonparametrics, and
usually a very good approach: Let $y|x$ be drawn from

\pause

\begin{description}
\item[] $N(m_1(x,\theta),\sigma^2_1(x,\theta))$ with probability
$q_1(x,\theta)$;
\item[] \ldots
\item[] $N(m_K(x,\theta),\sigma^2_K(x,\theta))$ with probability
$q_K(x,\theta).$
\end{description}
where you choose some parameterizations, and the $q_k$'s are positive
and sum to 1.

\pause

Can be estimated by maximum-likelihood:


\[
\max_{\theta} \sum_{i=1}^n  \log\left(
\sum_{k=1}^K
\frac{q_k(x_i,\theta)}{\sigma_k(x,\theta)}
\phi\left(\frac{y_i-m_k(x_i,\theta)}{\sigma_k(x_i,\theta)}\right)\right).
\]


\pause

Usually works very well with $K\leq 3$ (perhaps after transforming
$y$ to $\log y$, e.g).


}



 
\frame{\frametitle{Splines: trading off fit and smoothness}

\pause

Choose some $0<\lambda<\infty$ and
\[
\min_{m(.)} \sum_i (y_i-m(x_i))^2+\lambda J(m),
\]

\pause

Then we ``obtain'' the
natural cubic spline with knots=$(x_1,\ldots,x_n)$:

\pause

\begin{itemize}[<+->]
\item $m$ is a cubic polynomial between consecutive $x_i$'s
\item it is linear out-of-sample
\item it is $C^2$ everywhere. 
\end{itemize}

\pause

``Consecutive'' implies one-dimensional\ldots harder to generalize to
$p_x>1$.

\pause

\textbf{Orthogonal polynomials:} check out Chebyshev, $1, x,
2x^2-1,4x^3-3x\ldots$ (on $[-1,1]$ here.)
}


\begin{frame}
\frametitle{Additive models}
\pause
{\em Additive model:} $y=\alpha+\sum_{j=1}^p + f_j (X_j) + +\epsilon$\\
\pause
{\em Backfitting algorithm:} start with $\hat{a}=\overline{y}_n$, and some zero--mean
guesses  $\hat{f}_j \equiv 0 $.
\pause
Then for $j=1,\ldots,p,\ldots ,1,2,\ldots,p,\ldots$,
\pause
\begin{enumerate}[<+->]
\item   Define 
\begin{eqnarray*}
f_j &\leftarrow& S_j[\{y_i - \hat{\alpha} - \sum_{k\neq j} \hat{f}_k (x_{ik})\}_1^N ]\\
f_j &\leftarrow&  \hat{f}_j - \frac{1}{N} \sum_{i=1}^N \hat{f}_j (x_{ij}).
\end{eqnarray*}
\item Regress $\hat{y}$ on $x_j$ to get $R_j$; then replace $\hat{r}_j$ with $R_j-\frac{1}{n}\sum_i \hat{r}_j(x_{ji})$ (where $S_j$ is some cubic smoothing spline).
\item Iterate until $\hat{f}_j$ doesn't change.
\end{enumerate}
\end{frame}



\end{document}

